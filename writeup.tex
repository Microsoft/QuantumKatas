\documentclass[a4paper]{article}

\usepackage[english]{babel}
\usepackage[utf8x]{inputenc}
\usepackage{amsmath}
\usepackage{graphicx}
\usepackage[colorinlistoftodos]{todonotes}
\usepackage{diagbox}
\usepackage{amsfonts}
\DeclareMathOperator*{\argmax}{arg\,max}

\title{CSE 490 Q Final Project Report}
\author{Li Du, Han Luo, Liangyu Zhao}

\begin{document}
\maketitle
\section*{{Abstract}}

For the final project, our team of 3 decided to create Katas on the topic Quantum Fourier Transform (QFT). Quantum Katas, maintained by Microsoft, are open-sourced programming exercises for learning Q\# and quantum computing. Currently, topics covered include Basic Gates, Simon's algorithm, Grover's algorithm etc., except Quantum Fourier Transform. Therefore, we believe creating programming exercises on QFT complement the various topics that have been covered by the Katas already and help people who hope to learn about quantum computing and Q\#. We presents the exercises by breaking the implementation of QFT and its applications into multiple steps and provide test cases for user to verify their answers.

\end{document}

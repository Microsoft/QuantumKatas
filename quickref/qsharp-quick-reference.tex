% Copyright (c) Microsoft Corporation. All rights reserved.
% Licensed under the MIT License.

%!TEX program=xelatex
% This cheatsheet is based on the template
% provided at https://gist.github.com/alexander-yakushev/c773543bf9a957749f79.
\documentclass[10pt,english,landscape]{article}
\usepackage{multicol}
\usepackage{calc}
\usepackage[landscape]{geometry}
\usepackage{color,graphicx,overpic}
\usepackage{amsmath}

% Workaround for left brackets not rendering
% https://tex.stackexchange.com/questions/520847/left-matrix-bracket-not-rendered-properly

\newbox\matrixrepresentationright
\sbox\matrixrepresentationright{\parbox{\textwidth}{
  \footnotesize
  $\qquad \underrightarrow{n \text{ columns}} \\
  \rotatebox[origin=c]{270}{$\underrightarrow{m \text{ rows}}$}
  \begin{bmatrix} 
    a_{0,0}   & \cdots & a_{0,n-1} \\ 
    \vdots    & \ddots & \vdots \\
    a_{m-1,0} & \cdots & a_{m-1,n-1} 
  \end{bmatrix}$
}}

\newbox\vectorrepresentationright
\sbox\vectorrepresentationright{
  \footnotesize
  $\begin{bmatrix}
    a_0 \\ \vdots \\ a_{n-1}
  \end{bmatrix}$
}

\newbox\matrixadditionleft
\sbox\matrixadditionleft{\parbox{\textwidth}{
  \footnotesize
  Addition \linebreak
  $\begin{bmatrix} 
    a & b \\ 
    c & d 
    \end{bmatrix} + 
  \begin{bmatrix}
    e & f \\
    g & h
  \end{bmatrix}$ 
}}

\newbox\matrixadditionright
\sbox\matrixadditionright{\parbox{\textwidth}{
  \footnotesize
  $ \begin{bmatrix}  
    a + e & b + f \\
    c + g & d + h 
  \end{bmatrix} $
}}

\newbox\matrixscalarleft
\sbox\matrixscalarleft{\parbox{\textwidth}{
  \footnotesize
  Scalar product \newline
  $ a  \cdot \begin{bmatrix} 
    b & c \\
    d & e
  \end{bmatrix} $
}}

\newbox\matrixscalarright
\sbox\matrixscalarright{\parbox{\textwidth}{
  \footnotesize
  $ \begin{bmatrix}
    a \cdot b & a \cdot c \\
    a \cdot d & a \cdot e
  \end{bmatrix} $ 
}}

\newbox\matrixmultiplicationleft
\sbox\matrixmultiplicationleft{\parbox{\textwidth}{
  \footnotesize
  Matrix product \newline
  $ \begin{bmatrix} 
    a & b & c \\
    d & e & f
  \end{bmatrix} \cdot
  \begin{bmatrix} 
    x \\
    y \\
    z
  \end{bmatrix} $
}}

\newbox\matrixmultiplicationright
\sbox\matrixmultiplicationright{\parbox{\textwidth}{
  \footnotesize
  $ \begin{bmatrix} 
    a \cdot x + b \cdot y + c \cdot z \\
    d \cdot x + e \cdot y + f \cdot z 
  \end{bmatrix} $
}}

\newbox\matrixtransposeleft
\sbox\matrixtransposeleft{\parbox{\textwidth}{
  \footnotesize
  Transpose \newline 
  $ \begin{bmatrix}
    a & b & c\\
    d & e & f 
  \end{bmatrix}^T $
}}

\newbox\matrixtransposeright
\sbox\matrixtransposeright{\parbox{\textwidth}{
  \footnotesize
  $ \begin{bmatrix}
     a & d \\
    b & e \\
    c & f
  \end{bmatrix} $ 
}}

\newbox\matrixadjointleft
\sbox\matrixadjointleft{\parbox{\textwidth}{
  \footnotesize
  Adjoint \newline  
  $ \begin{bmatrix}
    a & b & c \\
    d & e & f
  \end{bmatrix}^\dagger $ 
}}

\newbox\matrixadjointright
\sbox\matrixadjointright{\parbox{\textwidth}{
  \footnotesize
  $ \begin{bmatrix}
     \overline{a} & \overline{e} \\
    \overline{b} & \overline{d} \\
     \overline{c} & \overline{f}
  \end{bmatrix} $ 
}}

\newbox\outerproductleft
\sbox\outerproductleft{\parbox{\textwidth}{
  \footnotesize
  Outer product \newline
  $ \begin{bmatrix} 
    a \\
    b
  \end{bmatrix} $  and $
  \begin{bmatrix} 
    x \\ 
    y \\
    z 
    \end{bmatrix} $
}}

\newbox\innerproductleft
\sbox\innerproductleft{\parbox{\textwidth}{
  \footnotesize
  Inner product \newline
  $ \left\langle 
    \begin{bmatrix}
      a \\
      b
    \end{bmatrix} ,
    \begin{bmatrix}  
      c \\
      d
    \end{bmatrix} 
  \right\rangle $ 
}}

\newbox\innerproductright
\sbox\innerproductright{\parbox{\textwidth}{
  \footnotesize
  $ \begin{bmatrix}
    a \\
    b
  \end{bmatrix}^\dagger
  \begin{bmatrix}
    c \\
    d
  \end{bmatrix} =
  \begin{bmatrix} 
    \overline{a} &  \overline{b} 
  \end{bmatrix}  
  \begin{bmatrix}
    c \\
    d
  \end{bmatrix} = 
  \overline{a} c + \overline{b} d $
}}

\newbox\outerproductright
\sbox\outerproductright{\parbox{\textwidth}{
  \footnotesize
  $ \begin{bmatrix}
    a \\
    b
  \end{bmatrix} 
  \begin{bmatrix}
    x \\
    y \\
    z
  \end{bmatrix}^\dagger =
  \begin{bmatrix}
    a \\
    b  
  \end{bmatrix}
  \begin{bmatrix}
    \overline{x} & \overline{y} & \overline{z}
  \end{bmatrix} = $ \newline
  $ = \begin{bmatrix} 
        a \cdot \overline{x} & a \cdot \overline{y} & a \cdot \overline{z} \\
        b \cdot \overline{x} & b \cdot \overline{y} & b \cdot \overline{z} \\
      \end{bmatrix} $ 
}}

\newbox\xgate
\sbox\xgate{\parbox{\textwidth}{
  \footnotesize
   $ \begin{bmatrix} 0 & 1 \\ 1 & 0 \end{bmatrix} $ 
}}

\newbox\ygate
\sbox\ygate{\parbox{\textwidth}{
  \footnotesize
  $ \begin{bmatrix} 0 & -i \\ i & 0 \end{bmatrix} $
}}

\newbox\zgate
\sbox\zgate{\parbox{\textwidth}{
  \footnotesize
  $ \begin{bmatrix} 1 & 0 \\ 0 & -1 \end{bmatrix} $
}}

\newbox\igate
\sbox\igate{\parbox{\textwidth}{
  \footnotesize
  $ \begin{bmatrix} 1 & 0 \\ 0 & 1 \end{bmatrix} $
}}

\newbox\hgate
\sbox\hgate{\parbox{\textwidth}{
  \footnotesize
  $ \frac{1}{\sqrt{2}}\begin{bmatrix} 1 & 1 \\ 1 & -1 \end{bmatrix} $
}}

\newbox\sgate
\sbox\sgate{\parbox{\textwidth}{
  \footnotesize
  $ \begin{bmatrix} 1 & 0 \\ 0 & i \end{bmatrix} $
}}

\newbox\tgate
\sbox\tgate{\parbox{\textwidth}{
  \footnotesize
   $ \begin{bmatrix} 1 & 0 \\ 0 & e^{i\pi/4} \end{bmatrix} $ 
}}

\newbox\rxgate
\sbox\rxgate{\parbox{\textwidth}{
  \footnotesize
   $\begin{bmatrix} \cos\frac{\theta}{2} & -i\sin\frac{\theta}{2} \\ -i\sin\frac{\theta}{2} &  \cos\frac{\theta}{2} \end{bmatrix} $ 
}}

\newbox\rygate
\sbox\rygate{\parbox{\textwidth}{
  \footnotesize
   $ \begin{bmatrix} \cos\frac{\theta}{2}  & -\sin\frac{\theta}{2} \\ \sin\frac{\theta}{2} & \cos\frac{\theta}{2} \end{bmatrix} $
}}

\newbox\rzgate
\sbox\rzgate{\parbox{\textwidth}{
  \footnotesize
  $ \begin{bmatrix} e^{-i\theta/2} & 0 \\ 0 & e^{i\theta/2} \end{bmatrix} $ 
}}


\newbox\rgate
\sbox\rgate{\parbox{\textwidth}{
  \footnotesize
   $ \begin{bmatrix} 1 & 0 \\ 0 & e^{i\theta} \end{bmatrix} $
}}

\newbox\cnotgate
\sbox\cnotgate{\parbox{\textwidth}{
  \footnotesize
   $ \begin{bmatrix} 1 & 0 & 0 & 0 \\ 0 & 1 & 0 & 0 \\ 0 & 0 & \mathbf{0} & \mathbf{1} \\ 0 & 0 & \mathbf{1} & \mathbf{0} \end{bmatrix} $
}}

\newbox\swapgate
\sbox\swapgate{\parbox{\textwidth}{
  \footnotesize
   $ \begin{bmatrix} 1 & 0 & 0 & 0 \\ 0 & \mathbf{0} & \mathbf{1} & 0 \\ 0 & \mathbf{1} & \mathbf{0} & 0 \\ 0 & 0 & 0 & 1 \end{bmatrix} $
}}

\newbox\controlledugate
\sbox\controlledugate{\parbox{\textwidth}{
  \footnotesize
  $\begin{bmatrix} 1 & 0 & 0 & 0 \\ 0 & 1 & 0 & 0 \\ 0 & 0 & a_{0,0} & a_{0,1} \\ 0 & 0 & a_{1,0} & a_{1,1} \end{bmatrix} $
}}

\newbox\ccnotgate
\sbox\ccnotgate{\parbox{\textwidth}{
  \footnotesize
  $\begin{bmatrix}
    1 & 0 & 0 & 0 & 0 & 0 & 0 & 0 \\
    0 & 1 & 0 & 0 & 0 & 0 & 0 & 0 \\
    0 & 0 & 1 & 0 & 0 & 0 & 0 & 0 \\
    0 & 0 & 0 & 1 & 0 & 0 & 0 & 0 \\
    0 & 0 & 0 & 0 & 1 & 0 & 0 & 0 \\
    0 & 0 &0  & 0 & 0 & 1 & 0 & 0 \\
    0 & 0 & 0 & 0 & 0 & 0 & \mathbf{0} & \mathbf{1} \\
    0 & 0 & 0 & 0 & 0 & 0 & \mathbf{1} & \mathbf{0}
\end{bmatrix}$
}}

\newbox\braketcx
\sbox\braketcx{\parbox{\textwidth}{
  \footnotesize
    $ |00\rangle\langle00| + |01\rangle\langle01| + |11\rangle\langle10| + |10\rangle\langle11| \newline \text{or} \newline |0\rangle\langle0| \otimes I + |1\rangle\langle1| \otimes X $
}}

\newbox\braketcu
\sbox\braketcu{\parbox{\textwidth}{
  \footnotesize
    $  |0\rangle\langle0| \otimes I + |1\rangle\langle1| \otimes U $
}}

\newbox\braketccx
\sbox\braketccx{\parbox{\textwidth}{
  \footnotesize
    $ |00\rangle\langle00| + |10\rangle\langle01| + a_{0,0}|10\rangle\langle10| + \newline a_{0,1}|10\rangle\langle11| + a_{1,0}|10\rangle\langle11| + a_{1,1}|11\rangle\langle11| \newline \text{Or} \newline |0\rangle\langle0| \otimes I + |1\rangle\langle1| \otimes U $
}}



% https://tex.stackexchange.com/a/192067
\usepackage{fontawesome}
  \newfontfamily{\FA}{[FontAwesome.otf]}

% Known issue in {fontspec}: https://github.com/wspr/fontspec/issues/312#issuecomment-342125206
% Since {fontspec} is included only when compiling with XeTeX, we guard
% our fix accordingly.
\usepackage{ifxetex}
    \ifxetex
        \let\latinencoding\relax
        \usepackage{fontspec}
        \setmainfont{Segoe UI}
        \setmonofont{Consolas}
    \fi

% \usepackage[T1]{fontenc}
% \usepackage[bitstream-charter]{mathdesign}
\usepackage[utf8]{inputenc}
\usepackage{url}
\usepackage{amsfonts}
\usepackage{array,booktabs}
\usepackage{textcomp}
\usepackage[usenames,dvipsnames,table]{xcolor}
\usepackage[most]{tcolorbox}
\usepackage{menukeys}
\usepackage{tabularx}
\usepackage{multirow}
\usepackage{colortbl}
\usepackage{tikz}
\usepackage{environ}
\usepackage{braket}

\usetikzlibrary{calc}
\pgfdeclarelayer{background}
\pgfdeclarelayer{foreground}
\pgfsetlayers{background,main,foreground}

\geometry{top=1cm,left=1cm,right=1cm,bottom=1cm}

\pagestyle{empty} % Turn off header and footer

% \renewcommand\rmdefault{phv} % Arial
% \renewcommand\sfdefault{phv} % Arial

% Redefine section commands to use less space
\makeatletter
\renewcommand{\section}{\@startsection{section}{1}{0mm}%
  {-1ex plus -.5ex minus -.2ex}%
  {0.5ex plus .2ex}%x
  {\normalfont\large\bfseries}}
\renewcommand{\subsection}{\@startsection{subsection}{2}{0mm}%
  {-1explus -.5ex minus -.2ex}%
  {0.5ex plus .2ex}%
  {\normalfont\normalsize\bfseries}}
\renewcommand{\subsubsection}{\@startsection{subsubsection}{3}{0mm}%
  {-1ex plus -.5ex minus -.2ex}%
  {1ex plus .2ex}%
  {\normalfont\small\bfseries}}
\makeatother

\setcounter{secnumdepth}{0} % Don't print section numbers
\setlength{\parindent}{0pt}
\setlength{\parskip}{0pt plus 0.5ex}

\definecolor{TableHead}{rgb}{0.353, 0.329, 0.667}
\definecolor{TableRow}{rgb}{0.816, 0.812, 0.902}

% Turn off warnings about slightly overfull boxes
% https://tex.stackexchange.com/questions/50830/do-i-have-to-care-about-bad-boxes/50850

\hfuzz=10pt

\NewEnviron{keysref}[1]{
  % \begin{center}
  \smallskip
  \begin{tikzpicture}
      \rowcolors{1}{}{TableRow}

      \node (tbl) [inner sep=0pt] {
        \begin{tabular}{p{2.5cm}p{5.05cm}}
          \rowcolor{TableHead}
          \multicolumn{2}{l}{\normalsize\textbf{\color{white}{#1}}}\parbox{0pt}{\rule{0pt}{0.3ex+\baselineskip}}\\
          \BODY
          \arrayrulecolor{TableHead}\specialrule{.17em}{0em}{.2em}
        \end{tabular}};
      \begin{pgfonlayer}{background}
        \draw[rounded corners=2pt,top color=TableHead,bottom color=TableHead, draw=white]
        ($(tbl.north west)-(0,-0.05)$) rectangle ($(tbl.north east)-(0.0,0.15)$);
        \draw[rounded corners=2pt,top color=TableHead,bottom color=TableHead, draw=white]
        ($(tbl.south west)-(0.0,-0.11)$) rectangle ($(tbl.south east)-(-0.0,-0.02)$);
      \end{pgfonlayer}
    \end{tikzpicture}
  % \end{center}
}

\NewEnviron{keysrefGates}[1]{
  % \begin{center}
  \smallskip
  \begin{tikzpicture}
      \rowcolors{1}{}{TableRow}

      \node (tbl) [inner sep=0pt] {
        \begin{tabular}{p{0.85cm}p{3.55cm}p{4.05cm}p{5.55cm}p{2.95cm}p{2.95cm}p{3.05cm}}
          \rowcolor{TableHead}
          \multicolumn{7}{l}{\normalsize\textbf{\color{white}{#1}}}\parbox{0pt}{\rule{0pt}{0.3ex+\baselineskip}}\\
          \BODY
          \arrayrulecolor{TableHead}\specialrule{.17em}{0em}{.2em}
        \end{tabular}};
      \begin{pgfonlayer}{background}
        \draw[rounded corners=2pt,top color=TableHead,bottom color=TableHead, draw=white]
        ($(tbl.north west)-(0,-0.05)$) rectangle ($(tbl.north east)-(0.0,0.15)$);
        \draw[rounded corners=2pt,top color=TableHead,bottom color=TableHead, draw=white]
        ($(tbl.south west)-(0.0,-0.11)$) rectangle ($(tbl.south east)-(-0.0,-0.02)$);
      \end{pgfonlayer}
    \end{tikzpicture}
  % \end{center}
}

\NewEnviron{keysrefMultiGates}[1]{
  % \begin{center}
  \smallskip
  \begin{tikzpicture}
      \rowcolors{1}{}{TableRow}

      \node (tbl) [inner sep=0pt] {
        \begin{tabular}{p{1.55cm}p{3,75cm}p{5,35cm}p{6,05cm}p{2.95cm}p{3.75cm}}
          \rowcolor{TableHead}
          \multicolumn{6}{l}{\normalsize\textbf{\color{white}{#1}}}\parbox{0pt}{\rule{0pt}{0.3ex+\baselineskip}}\\
          \BODY
          \arrayrulecolor{TableHead}\specialrule{.17em}{0em}{.2em}
        \end{tabular}};
      \begin{pgfonlayer}{background}
        \draw[rounded corners=2pt,top color=TableHead,bottom color=TableHead, draw=white]
        ($(tbl.north west)-(0,-0.05)$) rectangle ($(tbl.north east)-(0.0,0.15)$);
        \draw[rounded corners=2pt,top color=TableHead,bottom color=TableHead, draw=white]
        ($(tbl.south west)-(0.0,-0.11)$) rectangle ($(tbl.south east)-(-0.0,-0.02)$);
      \end{pgfonlayer}
    \end{tikzpicture}
  % \end{center}
}


% https://tex.stackexchange.com/a/102523
\newcommand{\forceindent}[1]{\leavevmode{\parindent=#1\indent}}

%% CUSTOM NOTATION %%

\newcommand{\qs}{Q\#}
\newcommand{\unixlike}{\hfill\faApple\faLinux}
\newcommand{\ctrllike}{\hfill\faWindows\faLinux}

%%%%%%%%%%%%%%%%%%%%%%%%%%%%%%%%%%%%%%%%%%%%%%%%%%%%%%%%%%%%%%%%%%%%%%%%%%%%%%



\begin{document}

\raggedright\

\Large{\qs~0.15 Language Quick Reference}

\footnotesize
\begin{multicols}{3}

  \begin{keysref}{Primitive Types}
    64-bit integers         & \texttt{Int} \\
    Double-precision \newline floats & \texttt{Double} \\
    Booleans                & \texttt{Bool}   \newline 
                              e.g.: \texttt{true} or \texttt{false} \\
    Qubits                  & \texttt{Qubit}  \\
    Pauli basis             & \texttt{Pauli}  \newline
                              e.g.: \texttt{PauliI}, \texttt{PauliX}, \texttt{PauliY}, or \texttt{PauliZ} \\
    Measurement \newline results     & \texttt{Result} \newline
                              e.g.: \texttt{Zero} or \texttt{One} \\
    Sequences of \newline integers   & \texttt{Range}  \newline
                              e.g.: \texttt{1..10} or \texttt{5..-1..0} \\
    Strings                 & \texttt{String} \newline 
                              e.g.: \texttt{"Hello Quantum!"} \\
    "Return no \newline information" type   & \texttt{Unit} \newline e.g.: \texttt{()} \\
  \end{keysref}

  \begin{keysref}{Derived Types}
    Arrays                  & \texttt{\emph{elementType}[]} \\
    Tuples                  & \texttt{(\emph{type0}, \emph{type1}, ...)} \newline
                              e.g.: \texttt{(Int, Qubit)} \\
    Functions               & \texttt{\emph{input} -> \emph{output}} \newline
                              e.g.: \texttt{ArcCos : (Double) -> Double} \\
    Operations              & \texttt{\emph{input} => \emph{output} is \emph{variants}} \newline
                              e.g.: \texttt{H : (Qubit => Unit is Adj)} \\
  \end{keysref}
  
  \begin{keysref}{User-Defined Types}
  Declare UDT with \newline anonymous items & \texttt{newtype \emph{Name} = (Type, Type);} \newline
                                     e.g.: \texttt{newtype \emph{Pair} = (Int, Int);} \\
  Define UDT literal               & \texttt{\emph{Name}(\emph{baseTupleLiteral})} \newline
                                     e.g.: \texttt{let origin = Pair(0, 0);} \\
  Unwrap operator \texttt{!} \newline (convert UDT to \newline underlying type) 
                                   & \texttt{\emph{VarName}!} \newline
                                     e.g.: \texttt{let originTuple = origin!;} \newline \hphantom{..........}(now \texttt{originTuple = (0, 0)}) \\
  Declare UDT with \newline named items     & \texttt{newtype \emph{Name} =}   \newline \texttt{\hphantom{....}(\emph{Name1}: Type, \emph{Name2}: Type);} \newline
                                     e.g.: \texttt{newtype Complex =} \newline \texttt{\hphantom{....}(Re : Double, Im : Double);} \\
  Accessing named \newline items of UDTs    & \texttt{\emph{VarName}::\emph{ItemName}} \newline
                                     e.g.: \texttt{complexVariable::Re} \\
  Update-and-\newline reassign for named \newline UDT items & \texttt{set \emph{VarName} w/= \emph{ItemName} <- \emph{val};} \newline
                                            e.g.: \texttt{mutable p = Complex(0., 0.);} \newline
                                                  \texttt{\hphantom{....}set p w/= Re <- 1.0;} \\
  \end{keysref}
  
  \begin{keysref}{Symbols and Variables}
    Declare immutable \newline symbol           & \texttt{let \emph{varName} = \emph{value}} \\
    Declare mutable \newline symbol (variable)  & \texttt{mutable \emph{varName} = \emph{initialValue}} \\
    Update mutable \newline symbol (variable)   & \texttt{set \emph{varName} = \emph{newValue}} \\
    Apply-and-reassign                          & \texttt{set \emph{varName} \emph{operator}= \emph{expression}} \newline
                                                  e.g.: \texttt{set counter += 1;} \\ 
  \end{keysref}

  \columnbreak

  \begin{keysref}{Functions and Operations}
    Define function \newline (classical routine)
              & \texttt{function \emph{Name}(\emph{in0} : \emph{type0}, ...) \newline
                : \emph{returnType} \{} \newline
                \texttt{\hphantom{....}// \emph{function body}} \newline
                \texttt{\}} \\
    Call function & \texttt{\emph{Name}(\emph{parameters})} \newline
                    e.g.: \texttt{let two = Sqrt(4.0);} \\
    Define operation \newline (quantum routine) \newline with explicitly \newline specified body, \newline controlled and \newline adjoint variants
               & \texttt{operation \emph{Name}(\emph{in0} : \emph{type0}, ...) \newline 
                 : \emph{returnType} \{} \newline
                 \texttt{\hphantom{....}body \{ ... \}} \newline
                 \texttt{\hphantom{....}adjoint \{ ... \}} \newline
                 \texttt{\hphantom{....}controlled \{ ... \}} \newline
                 \texttt{\hphantom{....}adjoint controlled \{ ... \}} \newline
                 \texttt{\}} \\
    Define operation \newline with automatically \newline generated adjoint \newline and controlled \newline variants
               & \texttt{operation \emph{Name}(\emph{in0} : \emph{type0}, ...) \newline 
                 : \emph{returnType} is Adj + Ctl \{} \newline
                 \texttt{\hphantom{....}... \newline
                 \} } \\
    Call operation & \texttt{\emph{Name}(\emph{parameters})} \newline
                     e.g.: \texttt{Ry(0.5 * PI(), q);} \\
    Call adjoint \newline operation & \texttt{Adjoint \emph{Name}(\emph{parameters})} \newline
                     e.g.: \texttt{Adjoint Ry(0.5 * PI(), q);} \\
    Call controlled \newline operation & \texttt{Controlled \emph{Name}(\emph{controlQubits}, \newline \hphantom{....}\emph{parameters})} \newline
                     e.g.: \texttt{Controlled Ry(controls, \newline \hphantom{....}(0.5 * PI(), target));} \\
  \end{keysref}

  \begin{keysref}{Control Flow}
    Iterate over \newline a range of numbers
                       & \texttt{for \emph{index} in \emph{range} \{ }\newline 
                         \texttt{\hphantom{....}// Use integer \emph{index}} \newline
                         \texttt{\hphantom{....}...} \newline
                         \texttt{\}} \newline
                         e.g.: \texttt{for i in 0..N-1 \{ ... \}} \\
    While loop \newline (within functions)
                       & \texttt{while (\emph{condition}) \{} \newline
                         \texttt{\hphantom{....}...} \newline
                         \texttt{\}} \\
    Iterate over \newline an array & \texttt{for \emph{val} in \emph{array} \{ }\newline 
                                     \texttt{\hphantom{....}// Use value \emph{val}} \newline
                                     \texttt{\hphantom{....}...} \newline
                                     \texttt{\}} \newline
                                     e.g.: \texttt{for q in register \{ ... \}} \\
    Repeat-until- \newline success loop      & \texttt{repeat \{ ... \} \newline until \emph{condition} \newline fixup \{ ... \}} \\
    Conditional \newline statement & \texttt{if \emph{cond1} \{ ... \}} \newline 
                                     \texttt{elif \emph{cond2} \{ ... \}} \newline
                                     \texttt{else \{ ... \}} \\
    Ternary operator   & \texttt{\emph{condition} ? \emph{caseTrue} | \emph{caseFalse} } \\
    Return a value     & \texttt{return \emph{value}} \\
    Stop with an error & \texttt{fail "\emph{Error message}"} \\
    Conjugations \newline ($ABA^\dagger$ pattern)
                       & \texttt{within \{ ... \}} \newline
                         \texttt{apply \{ ... \}} \\
  \end{keysref}
  
  \columnbreak

  \begin{keysref}{Arrays}
    Allocate array       & \texttt{mutable \emph{name} = new \emph{Type}[\emph{length}]} \newline
                           e.g.: \texttt{mutable b = new Bool[2];} \\
    Get array length     & \texttt{Length(\emph{name})} \\
    Access k-th element  & \texttt{\emph{name}[k]} \newline NB: indices are 0-based  \\
    Assign k-th element \newline (copy-and-update) 
                         & \texttt{set \emph{name} w/= \emph{k} <- \emph{value}} \newline
                           e.g.: \texttt{set b w/= 0 <- true;} \\
    Array literal        & \texttt{[\emph{value0}, \emph{value1}, ...]} \newline
                           e.g.: \texttt{let b = [true, false, true];} \\
    Array concatenation  & \texttt{\emph{array1} + \emph{array2}} \newline 
                           e.g.: \texttt{let t = [1, 2, 3] + [4, 5];} \\
    Slicing (subarray)   & \texttt{\emph{name}[\emph{sliceRange}]} \newline
                           e.g.: if \texttt{t = [1, 2, 3, 4, 5]}, then \newline
                           \texttt{\hphantom{..}t[1 .. 3]\hphantom{....}} is \texttt{[2, 3, 4]} \newline
                           \texttt{\hphantom{..}t[3 ...]\hphantom{.....}} is \texttt{[4, 5]} \newline
                           \texttt{\hphantom{..}t[... 1]\hphantom{.....}} is \texttt{[1, 2]} \newline
                           \texttt{\hphantom{..}t[0 .. 2 ...]\hphantom{}} is \texttt{[1, 3, 5]} \newline
                           \texttt{\hphantom{..}t[...-1...]\hphantom{..}} is \texttt{[5, 4, 3, 2, 1]} \\
  \end{keysref}

  \begin{keysref}{Debugging (classical)}
    Print a string     & \texttt{Message("Hello Quantum!")} \\
    Print an \newline interpolated string   
                       & \texttt{Message(\$"Value = \{\emph{val}\}")} \\
  \end{keysref}
  
  \section{Resources}
  
  \begin{keysref}{Documentation}
    Quantum \newline Development Kit & \url{https://docs.microsoft.com/azure/quantum} \\
    QDK user guides & \url{https://docs.microsoft.com/azure/quantum/user-guide} \\
    \qs~Libraries \newline Reference & \url{https://docs.microsoft.com/qsharp/api} \\
  \end{keysref}

  \begin{keysref}{\qs~Code Repositories}
    QDK Samples                                & \url{https://github.com/microsoft/quantum} \\
    QDK Libraries                              & \url{https://github.com/microsoft/QuantumLibraries} \\
    Quantum Katas \newline (tutorials)         & \url{https://github.com/microsoft/QuantumKatas} \\
    \qs~compiler and \newline extensions       & \url{https://github.com/microsoft/qsharp-compiler} \\
    Simulation \newline framework              & \url{https://github.com/microsoft/qsharp-runtime} \\
    Jupyter kernel and \newline Python host    & \url{https://github.com/microsoft/iqsharp} \\
    Source code for \newline the documentation & \url{https://github.com/MicrosoftDocs/quantum-docs} \\
  \end{keysref}

  \columnbreak

  \begin{keysref}{Qubit Allocation}
    Allocate a register \newline of $N$ qubits 
                         & \texttt{use \emph{reg} = Qubit[\emph{N}]; } \newline
                           \texttt{//} Qubits in \texttt{\emph{reg}} start in $\ket{0}$. \newline
                           \texttt{...} \newline
                           \texttt{//} Qubits must be returned to $\ket{0}$. \\
    Allocate one qubit   & \texttt{use \emph{one} = Qubit(); } \newline
    						 \texttt{...} \\
    Allocate a mix of \newline qubit registers and \newline individual qubits
                         & \texttt{use (\emph{x}, \emph{y}, ... ) = } \newline 
                           \texttt{\hphantom{...}(Qubit[N], Qubit(), ... ); } \newline
                           \texttt{...} \\
  \end{keysref}

  \begin{keysref}{Debugging (quantum)}
    Print amplitudes \newline of wave function               & \texttt{DumpMachine("dump.txt")} \\
    Assert that a qubit is in $\ket{0}$ or $\ket{1}$ state   & \texttt{AssertQubit(Zero, \emph{zeroQubit})} \newline
                                                               \texttt{AssertQubit(One, \emph{oneQubit})} \\
  \end{keysref}

  


 
  \begin{keysref}{Measurements}
    Measure qubit in \newline Pauli $Z$ basis & \texttt{M(\emph{oneQubit})} \newline yields a \texttt{Result} (\texttt{Zero} or \texttt{One}) \\
    Reset qubit to $\ket{0}$     & \texttt{Reset(\emph{oneQubit})} \\
    Reset an array of \newline qubits to $\ket{0..0}$ & \texttt{ResetAll(\emph{register})} \\
  \end{keysref}

  

  \section{Working with \qs~from command line}

  \begin{keysref}{Command Line Basics}
    Change directory    & \texttt{cd \emph{dirname}} \\
    Go to home          & \texttt{cd \textasciitilde} \\
    Go up one directory & \texttt{cd ..} \\
    Make new directory  & \texttt{mkdir \emph{dirname}} \\
    Open current \newline directory in VS Code & \texttt{code .} \\
  \end{keysref}

  \begin{keysref}{Working with \qs~Projects}
    Create new project & \texttt{dotnet new console -lang Q\# \newline 
                         --output \emph{project-dir}} \\
    Change directory to \newline project directory & \texttt{cd \emph{project-dir}} \\
    Build project                     & \texttt{dotnet build} \\
    Run all unit tests                & \texttt{dotnet test} \\
  \end{keysref}

  \columnbreak
  
  \section{Math reference}

  \begin{keysref}{Complex Arithmetic}
    $i^2 $                         & $-1$ \\
    $ (a + bi) + (c + di) $        & $ (a + c) + (b + d)i $ \\
    $ (a + bi)  (c + di) $         & $ a \cdot c + a \cdot di + b \cdot ci + (b \cdot d)i^2 =$ \newline
                                     $ = (a \cdot c - b \cdot d) + (a \cdot d + b \cdot c)i $ \\
	  Complex conjugate              & $ \overline{a + bi} = a - bi $ \\
    Division $ \frac{a+bi}{c+di} $ & $ \frac{a+bi}{c+di} \cdot 1 = \frac{a+bi}{c+di} \cdot \frac{c-di}{c-di} = \frac{(a+bi)(c+di)}{c^2 + d^2} $ \\
    Modulus $|a + bi|$             & $ \sqrt{a^2 + b^2} $ \\
    $e^{i\theta} $                 & $ \cos \theta + i \sin \theta $ \\ 
    $ e^{a + bi} $                 & $ e^a \cdot e^{bi} = e^a \cos b  + i e^a \sin b $  \\
    $ r^{a+bi} $                   & $ r^a \cdot r^{bi} = r^a \cdot e^{bi\ln r} = $ \newline 
                                     $ = r^a \cos(b\ln r) + i \cdot r^a \sin(b\ln r) $ \\
    Polar form $re^{i\theta}$ to \newline Cartesian form $a + bi$ & $ a = r \cos \theta \newline b = r \sin \theta $ \\
    Cartesian form $a + bi$ \newline to polar form $re^{i\theta}$ & $ r = \sqrt{a^2 + b^2}  \newline  \theta = \arctan(\frac{b}{a}) $ \\
  \end{keysref}

  \begin{keysref}{Linear Algebra}

    $m \times n$ matrix & \usebox\matrixrepresentationright \\

    Vector of size $n$ & \usebox\vectorrepresentationright \\
    
    \usebox\matrixadditionleft & \usebox\matrixadditionright \\
    
    \usebox\matrixscalarleft &\usebox\matrixscalarright \\
    
    \usebox\matrixmultiplicationleft & \usebox\matrixmultiplicationright \\

    \usebox\matrixtransposeleft & \usebox\matrixtransposeright \\   
    
    \usebox\matrixadjointleft & \usebox\matrixadjointright \\

    \usebox\innerproductleft & \usebox\innerproductright \\   
    
    \usebox\outerproductleft & \usebox\outerproductright \\
  
  \end{keysref}
  
  \newpage
  \end{multicols}
  \section{Gates reference}

  \begin{keysrefGates}{Single Qubit gates}

    Gate & 
    Matrix representation  & 
    Ket-bra representation & 
    Applying to $ \ket{\psi} = \alpha\ket{0} + \beta\ket{1} $ & 
    Applying to basis states: & $\ket{0},\: \ket{1},\: \ket{+},\: \ket{-}\:$ and  & 
    $\ket{\pm i} = \frac{1}{\sqrt2}(\ket{0} \pm i\ket{1})$ \\
    
    X & 
    \usebox\xgate & $ \ket{0}\bra{1} + \ket{1}\bra{0}$ & 
    $ \ket{\psi} = \alpha\ket{1} + \beta\ket{0} $ & 
    $ X\ket{0} = \ket{1} \newline X\ket{1} = \ket{0} $ & 
    $ X\ket{+} = \ket{+} \newline   X\ket{-} = -\ket{-}  $ &
    $ X\ket{i} = i\ket{-i} \newline X\ket{-i} = -i\ket{i} $    \\

    Y & 
    \usebox\ygate & 
    $ i(\ket{1}\bra{0} - \ket{0}\bra{1})$ & 
    $ Y\ket{\psi} = i\big(\alpha\ket{1} - \beta\ket{0}\big)$ & 
    $Y\ket{0} = i\ket{1} \newline Y\ket{1} = -i\ket{0} $ & 
    $ Y\ket{+} = -i\ket{-}  \newline Y\ket{-} = i\ket{+} $ & 
    $ Y\ket{i} = \ket{i} \newline Y\ket{-i} = -\ket{-i}$   \\

    Z & 
    \usebox\zgate & 
    $ \ket{0}\bra{0} - \ket{1}\bra{1} $ & 
    $ Z\ket{\psi} = \alpha\ket{0} - \beta\ket{1} $ & 
    $ Z\ket{0} = \ket{0} \newline Z\ket{1} = -\ket{1} $ & 
    $  Z\ket{+} = \ket{-} \newline Z\ket{-} = \ket{+} $ & 
    $ Z\ket{i} = \ket{-i} \newline Z\ket{-i} = \ket{i} $  \\

    I & 
    \usebox\igate & 
    $ \ket{0}\bra{0} + \ket{1}\bra{1} $  & 
    $ I\ket{\psi} = \ket{\psi} $ & & & \\

    H & 
    \usebox\hgate & 
    $ \ket{0}\bra{+} + \ket{1}\bra{-} $ & 
    $ H\ket{\psi} = \alpha\ket{+} + \beta\ket{-} = \frac{\alpha + \beta}{\sqrt{2}}\ket{0} + \frac{\alpha - \beta}{\sqrt{2}}\ket{1} $ & 
    $ H\ket{0} = \ket{+}   \newline H\ket{1} = \ket{-} $ & 
    $ H\ket{+} = \ket{0} \newline H\ket{-} = \ket{1} $ & 
    $ H\ket{i} = e^{i\pi/4}\ket{-i} \newline H\ket{-i} = e^{-i\pi/4}\ket{i} $ \\

    S & 
    \usebox\sgate & 
    $ \ket{0}\bra{0} + i\ket{1}\bra{1} $ & 
    $ S\ket{\psi} = \alpha\ket{0} + i\beta\ket{1} $ & 
    $ S\ket{0} = \ket{0} \newline S\ket{1} = i\ket{1} $ & 
    $ S\ket{+} = \ket{i} \newline S\ket{-} = \ket{-i} $ & 
    $ S\ket{i} = \ket{-} \newline S\ket{-i} = \ket{+} $  \\

    T & 
    \usebox\tgate & 
    $ \ket{0}\bra{0} + e^{i\pi/4}\ket{1} \bra{1} $ & 
    $ T\ket{\psi} = \alpha\ket{0} + e^{i\pi/4} \beta \ket{1} $ & 
    $ T\ket{0} = \ket{0}  $ &
    $ T\ket{1} = e^{i\pi/4}\ket{1} $ &  \\

    $ R_x(\theta) $ & 
    \usebox\rxgate & 
    $ \cos\frac{\theta}{2}\ket{0}\bra{0}  -i\sin\frac{\theta}{2}\ket{1}\bra{0} - \newline 
      -i\sin\frac{\theta}{2}\ket{0} \bra{1} + \cos\frac{\theta}{2}\ket{1}\bra{1} $ & 
    $ R_x(\theta)\ket{\psi} = \ (\alpha\cos\frac{\theta}{2} - i\beta\sin\frac{\theta}{2})\ket{0} + \newline 
      + (\beta\cos\frac{\theta}{2} - i\alpha\sin\frac{\theta}{2})\ket{1} $ & 
    $ R_x(\theta)\ket{0} = \newline
      = \cos\frac{\theta}{2}\ket{0} - i\sin\frac{\theta}{2}\ket{1} $ & 
    $ R_x(\theta)\ket{1} = \newline
      = \cos\frac{\theta}{2}\ket{1} - i\sin\frac{\theta}{2}\ket{0} $ & \\ 

    $ R_y(\theta) $ & 
    \usebox\rygate   & 
    $ \cos\frac{\theta}{2}\ket{0}\bra{0} + \sin\frac{\theta}{2} \ket{1}\bra{0} - \newline 
      -\sin\frac{\theta}{2}\ket{0}\bra{1} + \cos\frac{\theta}{2}\ket{1}\bra{1}$ & 
    $ R_y(\theta)\ket{\psi} =  (\alpha\cos\frac{\theta}{2} - \beta\sin\frac{\theta}{2})\ket{0} + \newline 
      + (\beta\cos\frac{\theta}{2} + \alpha\sin\frac{\theta}{2})\ket{1} $ & 
    $ R_y(\theta)\ket{0} = \newline 
      = \cos\frac{\theta}{2}\ket{0} + \sin\frac{\theta}{2}\ket{1} $ & 
    $ R_y(\theta)\ket{1} = \newline 
      = \cos\frac{\theta}{2}\ket{1} - \sin\frac{\theta}{2}\ket{0} $  &  \\

    $ R_z(\theta) $ & \usebox\rzgate & 
    $  e^{-i\theta/2}\ket{0}\bra{0} + e^{i\theta/2}\ket{1} \bra{1} $ & 
    $ R_z(\theta)\ket{\psi} = \alpha e^{-i\theta/2}\ket{0} + \beta e^{i\theta/2}\ket{1} $ & 
    $ R_z(\theta)\ket{0} = e^{-i\theta/2}\ket{0} $ & 
    $ R_z(\theta)\ket{1} = e^{i\theta/2}\ket{1} $  &  \\

    $ R_1(\theta) $ & 
    \usebox\rgate & 
    $  \ket{0}\bra{0} + e^{i\theta}\ket{1} \bra{1} $ & 
    $ R_1(\theta)\ket{\psi} =  \alpha\ket{0} + \beta e^{i\theta}\ket{1} $ &  
    $ R_1(\theta)\ket{0} = \ket{0} $ & 
    $ R_1(\theta)\ket{1} = e^{i\theta}\ket{1} $  &  \\

  \end{keysrefGates}
  
  \begin{keysrefMultiGates}{Two-qubit gates}
  
    Gate &
    Matrix Representation & 
    Ket-Bra Representation &
    Applying to $\ket{\psi} = \alpha\ket{00} + \beta\ket{01} + \gamma\ket{10} + \delta\ket{11}$ &
    Applying to basis states & \\
    
    CNOT &
    \usebox\cnotgate & 
    \usebox\braketcx &
    $ \text{CNOT}\ket{\psi} = \alpha\ket{00} + \beta\ket{01} + $ {\boldmath $ \delta\ket{10} + \gamma\ket{11} $} &
    $ \text{CNOT}\ket{00} = \ket{00} \newline
      \text{CNOT}\ket{01} = \ket{01} $ &
    $ \text{CNOT}\ket{10} = \ket{11} \newline
      \text{CNOT}\ket{11} = \ket{10} $ \\
 
    SWAP &
    \usebox\swapgate & 
    $ \ket{00}\bra{00} + \ket{01}\bra{10} + \ket{10}\bra{01} + \ket{11}\bra{11} $ &
    $ \text{SWAP}\ket{\psi} = \alpha\ket{00} + $ {\boldmath $ \gamma\ket{01} + \beta\ket{10} $} $+ \delta\ket{11}$ &
    $ \text{SWAP}\ket{00} = \ket{00} \newline
      \text{SWAP}\ket{01} = \ket{10} $ &
    $\text{SWAP}\ket{10} = \ket{01} \newline
      \text{SWAP}\ket{11} = \ket{11} $ \\
      
    Controlled U &
    \usebox\controlledugate  & 
    \usebox\braketcu & 
    $ \text{CU} \ket{\psi} =  \alpha\ket{00} + \beta\ket{01} + \newline 
      + (\gamma  a_{0,0} + \delta  a_{0,1}) \ket{10} + (\gamma  a_{1,0} + \delta  a_{1,1}) \ket{11}  $ & 
    $ \text{CU} \ket{00} = \ket{00} \newline
      \text{CU} \ket{01} = \ket{01} $ & 
    $\text{CU} \ket{10} = a_{0,0} \ket{10} + a_{1,0} \ket{11} \newline 
     \text{CU} \ket{11} = a_{0,1} \ket{10} + a_{1,1} \ket{11} $ \\
  \end{keysrefMultiGates}
  
  \begin{keysrefMultiGates}{Toffoli (CCNOT) gate} 
  
    Gate &
    Matrix Representation &
    Ket-Bra Representation &
    Applying to $\ket{\psi} = \alpha\ket{000} + \beta\ket{001} + \gamma\ket{010} + \delta\ket{011} + \epsilon\ket{100} + \lambda\ket{101}  + \eta\ket{110} + \kappa\ket{111} $ &
    Applying to basis states & \\
    
    
    CCNOT &
    \usebox\ccnotgate & 
    $ (I_2 - \ket{11}\bra{11}) \otimes I_1 + \ket{11}\bra{11} \otimes X $ &  %$ \ket{000}\bra{000} + \ket{001}\bra{001} + \ket{010}\bra{010} + \ket{011}\bra{011} + \ket{100}\bra{100} + \ket{101}\bra{101} + \ket{110}\bra{111} + \ket{111}\bra{110} $ &
    $ \text{CCNOT} \ket{\psi} = \alpha\ket{000} + \beta\ket{001} + \gamma\ket{010} + \delta\ket{011} + \epsilon\ket{100} + \lambda\ket{101}  + $ \boldmath$ \kappa \ket{110} + \eta\ket{111} $ &
    $ \text{CCNOT} \ket{000} = \ket{000} \newline \text{CCNOT} \ket{001} = \ket{001} 
      \newline \text{CCNOT} \ket{010} = \ket{010} \newline \text{CCNOT} \ket{011} = \ket{011} $ &
     $ \text{CCNOT} \ket{100} = \ket{100} \newline \text{CCNOT} \ket{101} = \ket{101} \newline 
       \text{CCNOT} \ket{110} = \ket{111} \newline \text{CCNOT} \ket{111} = \ket{110} $\\
  
  \end{keysrefMultiGates}

\end{document} 
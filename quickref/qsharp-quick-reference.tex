% Copyright (c) Microsoft Corporation. All rights reserved.
% Licensed under the MIT License.

%!TEX program=xelatex
% This cheatsheet is based on the template
% provided at https://gist.github.com/alexander-yakushev/c773543bf9a957749f79.
\documentclass[10pt,english,landscape]{article}
\usepackage{multicol}
\usepackage{calc}
\usepackage[landscape]{geometry}
\usepackage{color,graphicx,overpic}

% https://tex.stackexchange.com/a/192067
\usepackage{fontawesome}
  \newfontfamily{\FA}{[FontAwesome.otf]}

% Known issue in {fontspec}: https://github.com/wspr/fontspec/issues/312#issuecomment-342125206
% Since {fontspec} is included only when compiling with XeTeX, we guard
% our fix accordingly.
\usepackage{ifxetex}
    \ifxetex
        \let\latinencoding\relax
        \usepackage{fontspec}
        \setmainfont{Segoe UI}
        \setmonofont{Consolas}
    \fi

% \usepackage[T1]{fontenc}
% \usepackage[bitstream-charter]{mathdesign}
\usepackage[utf8]{inputenc}
\usepackage{url}
\usepackage{amsfonts}
\usepackage{array,booktabs}
\usepackage{textcomp}
\usepackage[usenames,dvipsnames,table]{xcolor}
\usepackage[most]{tcolorbox}
\usepackage{menukeys}
\usepackage{tabularx}
\usepackage{multirow}
\usepackage{colortbl}
\usepackage{tikz}
\usepackage{environ}
\usepackage{braket}

\usetikzlibrary{calc}
\pgfdeclarelayer{background}
\pgfdeclarelayer{foreground}
\pgfsetlayers{background,main,foreground}

\geometry{top=1cm,left=1cm,right=1cm,bottom=1cm}

\pagestyle{empty} % Turn off header and footer

% \renewcommand\rmdefault{phv} % Arial
% \renewcommand\sfdefault{phv} % Arial

% Redefine section commands to use less space
\makeatletter
\renewcommand{\section}{\@startsection{section}{1}{0mm}%
  {-1ex plus -.5ex minus -.2ex}%
  {0.5ex plus .2ex}%x
  {\normalfont\large\bfseries}}
\renewcommand{\subsection}{\@startsection{subsection}{2}{0mm}%
  {-1explus -.5ex minus -.2ex}%
  {0.5ex plus .2ex}%
  {\normalfont\normalsize\bfseries}}
\renewcommand{\subsubsection}{\@startsection{subsubsection}{3}{0mm}%
  {-1ex plus -.5ex minus -.2ex}%
  {1ex plus .2ex}%
  {\normalfont\small\bfseries}}
\makeatother

\setcounter{secnumdepth}{0} % Don't print section numbers
\setlength{\parindent}{0pt}
\setlength{\parskip}{0pt plus 0.5ex}

\definecolor{TableHead}{rgb}{0.353, 0.329, 0.667}
\definecolor{TableRow}{rgb}{0.816, 0.812, 0.902}

\NewEnviron{keysref}[1]{
  % \begin{center}
  \smallskip
  \begin{tikzpicture}
      \rowcolors{1}{}{TableRow}

      \node (tbl) [inner sep=0pt] {
        \begin{tabular}{p{2.5cm}p{5.05cm}}
          \rowcolor{TableHead}
          \multicolumn{2}{l}{\normalsize\textbf{\color{white}{#1}}}\parbox{0pt}{\rule{0pt}{0.3ex+\baselineskip}}\\
          \BODY
          \arrayrulecolor{TableHead}\specialrule{.17em}{0em}{.2em}
        \end{tabular}};
      \begin{pgfonlayer}{background}
        \draw[rounded corners=2pt,top color=TableHead,bottom color=TableHead, draw=white]
        ($(tbl.north west)-(0,-0.05)$) rectangle ($(tbl.north east)-(0.0,0.15)$);
        \draw[rounded corners=2pt,top color=TableHead,bottom color=TableHead, draw=white]
        ($(tbl.south west)-(0.0,-0.11)$) rectangle ($(tbl.south east)-(-0.0,-0.02)$);
      \end{pgfonlayer}
    \end{tikzpicture}
  % \end{center}
}

% https://tex.stackexchange.com/a/102523
\newcommand{\forceindent}[1]{\leavevmode{\parindent=#1\indent}}

%% CUSTOM NOTATION %%

\newcommand{\qs}{Q\#}
\newcommand{\unixlike}{\hfill\faApple\faLinux}
\newcommand{\ctrllike}{\hfill\faWindows\faLinux}

%%%%%%%%%%%%%%%%%%%%%%%%%%%%%%%%%%%%%%%%%%%%%%%%%%%%%%%%%%%%%%%%%%%%%%%%%%%%%%

\begin{document}

\raggedright\

\Large{\qs~0.10 Language Quick Reference}

\footnotesize
\begin{multicols}{3}

  \begin{keysref}{Primitive Types}
    64-bit integers         & \texttt{Int} \\
    Double-precision floats & \texttt{Double} \\
    Booleans                & \texttt{Bool}   \newline 
                              e.g.: \texttt{true} or \texttt{false} \\
    Qubits                  & \texttt{Qubit}  \\
    Pauli basis             & \texttt{Pauli}  \newline
                              e.g.: \texttt{PauliI}, \texttt{PauliX}, \texttt{PauliY}, or \texttt{PauliZ} \\
    Measurement \newline results     & \texttt{Result} \newline
                              e.g.: \texttt{Zero} or \texttt{One} \\
    Sequences of \newline integers   & \texttt{Range}  \newline
                              e.g.: \texttt{1..10} or \texttt{5..-1..0} \\
    Strings                 & \texttt{String} \newline 
                              e.g.: \texttt{"Hello Quantum!"} \\
    "Return no \newline information" type   & \texttt{Unit} \newline e.g.: \texttt{()} \\
  \end{keysref}

  \begin{keysref}{Derived Types}
    Arrays                  & \texttt{\emph{elementType}[]} \\
    Tuples                  & \texttt{(\emph{type0}, \emph{type1}, ...)} \newline
                              e.g.: \texttt{(Int, Qubit)} \\
    Functions               & \texttt{\emph{input} -> \emph{output}} \newline
                              e.g.: \texttt{ArcCos : (Double) -> Double} \\
    Operations              & \texttt{\emph{input} => \emph{output} is \emph{variants}} \newline
                              e.g.: \texttt{H : (Qubit => Unit is Adj)} \\
  \end{keysref}
  
  \begin{keysref}{User-Defined Types}
  Declare UDT with anonymous items & \texttt{newtype \emph{Name} = (Type, Type);} \newline
                                     e.g.: \texttt{newtype \emph{Pair} = (Int, Int);} \\
  Define UDT literal               & \texttt{\emph{Name}(\emph{baseTupleLiteral})} \newline
                                     e.g.: \texttt{let origin = Pair(0, 0);} \\
  Unwrap operator \texttt{!} \newline (convert UDT to \newline underlying type) 
                                   & \texttt{\emph{VarName}!} \newline
                                     e.g.: \texttt{let originTuple = origin!;} \newline \hphantom{..........}(now \texttt{originTuple = (0, 0)}) \\
  Declare UDT with named items     & \texttt{newtype \emph{Name} =}   \newline \texttt{\hphantom{....}(\emph{Name1}: Type, \emph{Name2}: Type);} \newline
                                     e.g.: \texttt{newtype Complex =} \newline \texttt{\hphantom{....}(Re : Double, Im : Double);} \\
  Accessing named items of UDTs    & \texttt{\emph{VarName}::\emph{ItemName}} \newline
                                     e.g.: \texttt{complexVariable::Re} \\
  Update-and-reassign for named UDT items & \texttt{set \emph{VarName} w/= \emph{ItemName} <- \emph{val};} \newline
                                            e.g.: \texttt{mutable p = Complex(0., 0.);} \newline
                                                  \texttt{\hphantom{....}set p w/= Re <- 1.0;} \\
  \end{keysref}
  
  \begin{keysref}{Symbols and Variables}
    Declare immutable \newline symbol           & \texttt{let \emph{varName} = \emph{value}} \\
    Declare mutable \newline symbol (variable)  & \texttt{mutable \emph{varName} = \emph{initialValue}} \\
    Update mutable \newline symbol (variable)   & \texttt{set \emph{varName} = \emph{newValue}} \\
    Apply-and-reassign                          & \texttt{set \emph{varName} \emph{operator}= \emph{expression}} \newline
                                                  e.g.: \texttt{set counter += 1;} \\ 
  \end{keysref}

  \columnbreak

  \begin{keysref}{Functions and Operations}
    Define function \newline (classical routine)
              & \texttt{function \emph{Name}(\emph{in0} : \emph{type0}, ...) : \emph{returnType} \{} \newline
                \texttt{\hphantom{....}// \emph{function body}} \newline
                \texttt{\}} \\
    Call function & \texttt{\emph{Name}(\emph{parameters})} \newline
                    e.g.: \texttt{let two = Sqrt(4.0);} \\
    Define operation \newline (quantum routine) \newline with explicitly \newline specified body, \newline controlled and \newline adjoint variants
               & \texttt{operation \emph{Name}(\emph{in0} : \emph{type0}, ...) : \emph{returnType} \{} \newline
                 \texttt{\hphantom{....}body \{ ... \}} \newline
                 \texttt{\hphantom{....}adjoint \{ ... \}} \newline
                 \texttt{\hphantom{....}controlled \{ ... \}} \newline
                 \texttt{\hphantom{....}adjoint controlled \{ ... \}} \newline
                 \texttt{\}} \\
    Define operation \newline with automatically \newline generated adjoint \newline and controlled \newline variants
               & \texttt{operation \emph{Name}(\emph{in0} : \emph{type0}, ...) : \emph{returnType} is Adj + Ctl \{} \newline
                 \texttt{\hphantom{....}... \newline
                 \} } \\
    Call operation & \texttt{\emph{Name}(\emph{parameters})} \newline
                     e.g.: \texttt{Ry(0.5 * PI(), q);} \\
    Call adjoint \newline operation & \texttt{Adjoint \emph{Name}(\emph{parameters})} \newline
                     e.g.: \texttt{Adjoint Ry(0.5 * PI(), q);} \\
    Call controlled \newline operation & \texttt{Controlled \emph{Name}(\emph{controlQubits}, \newline \hphantom{....}\emph{parameters})} \newline
                     e.g.: \texttt{Controlled Ry(controls, \newline \hphantom{....}(0.5 * PI(), target));} \\
  \end{keysref}

  \begin{keysref}{Control Flow}
    Iterate over \newline a range of numbers
                       & \texttt{for (\emph{index} in \emph{range}) \{ }\newline 
                         \texttt{\hphantom{....}// Use integer \emph{index}} \newline
                         \texttt{\hphantom{....}...} \newline
                         \texttt{\}} \newline
                         e.g.: \texttt{for (i in 0..N-1) \{ ... \}} \\
    While loop \newline (within functions)
                       & \texttt{while (\emph{condition}) \{} \newline
                         \texttt{\hphantom{....}...} \newline
                         \texttt{\}} \\
    Iterate over \newline an array & \texttt{for (\emph{val} in \emph{array}) \{ }\newline 
                                     \texttt{\hphantom{....}// Use value \emph{val}} \newline
                                     \texttt{\hphantom{....}...} \newline
                                     \texttt{\}} \newline
                                     e.g.: \texttt{for (q in register) \{ ... \}} \\
    Repeat-until-success loop      & \texttt{repeat \{ ... \} \newline until (\emph{condition}) \newline fixup \{ ... \}} \\
    Conditional \newline statement & \texttt{if (\emph{cond1}) \{ ... \}} \newline 
                                     \texttt{elif (\emph{cond2}) \{ ... \}} \newline
                                     \texttt{else \{ ... \}} \\
    Ternary operator   & \texttt{\emph{condition} ? \emph{caseTrue} | \emph{caseFalse} } \\
    Return a value     & \texttt{return \emph{value}} \\
    Stop with an error & \texttt{fail "\emph{Error message}"} \\
    Conjugations \newline ($ABA^\dagger$ pattern)
                       & \texttt{within \{ ... \}} \newline
                         \texttt{apply \{ ... \}} \\
  \end{keysref}
  
  \columnbreak

  \begin{keysref}{Arrays}
    Allocate array       & \texttt{mutable \emph{name} = new \emph{Type}[\emph{length}]} \
                           e.g.: \texttt{mutable b = new Bool[2];} \\
    Get array length     & \texttt{Length(\emph{name})} \\
    Access k-th element  & \texttt{\emph{name}[k]} \newline NB: indices are 0-based  \\
    Assign k-th element \newline (copy-and-update) 
                         & \texttt{set \emph{name} w/= \emph{k} <- \emph{value}} \newline
                           e.g.: \texttt{set b w/= 0 <- true;} \\
    Array literal        & \texttt{[\emph{value0}, \emph{value1}, ...]} \newline
                           e.g.: \texttt{let b = [true, false, true];} \\
    Array concatenation  & \texttt{\emph{array1} + \emph{array2}} \newline 
                           e.g.: \texttt{let t = [1, 2, 3] + [4, 5];} \\
    Slicing (subarray)   & \texttt{\emph{name}[\emph{sliceRange}]} \newline
                           e.g.: if \texttt{t = [1, 2, 3, 4, 5]}, then \newline
                           \texttt{\hphantom{..}t[1 .. 3]\hphantom{....}} is \texttt{[2, 3, 4]} \newline
                           \texttt{\hphantom{..}t[3 ...]\hphantom{.....}} is \texttt{[4, 5]} \newline
                           \texttt{\hphantom{..}t[... 1]\hphantom{.....}} is \texttt{[1, 2]} \newline
                           \texttt{\hphantom{..}t[0 .. 2 ...]\hphantom{}} is \texttt{[1, 3, 5]} \newline
                           \texttt{\hphantom{..}t[...-1...]\hphantom{..}} is \texttt{[5, 4, 3, 2, 1]} \\
  \end{keysref}

  \begin{keysref}{Debugging (classical)}
    Print a string     & \texttt{Message("Hello Quantum!")} \\
    Print an \newline interpolated string   
                       & \texttt{Message(\$"Value = \{\emph{val}\}")} \\
  \end{keysref}
  
  \section{Resources}
  
  \begin{keysref}{Documentation}
    Quantum \newline Development Kit & \url{https://docs.microsoft.com/quantum} \\
    \qs~Language \newline Reference  & \url{https://docs.microsoft.com/quantum/language} \\
    \qs~Libraries \newline Reference & \url{https://docs.microsoft.com/qsharp/api} \\
  \end{keysref}

  \begin{keysref}{\qs~Code Repositories}
    QDK Samples                                & \url{https://github.com/microsoft/quantum} \\
    QDK Libraries                              & \url{https://github.com/microsoft/QuantumLibraries} \\
    Quantum Katas \newline (tutorials)         & \url{https://github.com/microsoft/QuantumKatas} \\
    \qs~compiler and \newline extensions       & \url{https://github.com/microsoft/qsharp-compiler} \\
    Simulation \newline framework              & \url{https://github.com/microsoft/qsharp-runtime} \\
    Jupyter kernel and \newline Python host    & \url{https://github.com/microsoft/iqsharp} \\
    Source code for \newline the documentation & \url{https://github.com/MicrosoftDocs/quantum-docs-pr} \\
  \end{keysref}

  \columnbreak

  \begin{keysref}{Qubit Allocation}
    Allocate a register \newline of $N$ qubits 
                         & \texttt{using (\emph{reg} = Qubit[\emph{N}]) \{} \newline
                           \texttt{\hphantom{....}//} Qubits in \texttt{\emph{reg}} start in $\ket{0}$. \newline
                           \texttt{\hphantom{....}...} \newline
                           \texttt{\hphantom{....}//} Qubits must be returned to $\ket{0}$. \newline
                           \texttt{\}} \\
    Allocate one qubit   & \texttt{using (\emph{one} = Qubit()) \{ ... \} } \\
    Allocate a mix of \newline qubit registers and \newline individual qubits
                         & \texttt{using ((\emph{x}, \emph{y}, ... ) = } \newline 
                           \texttt{\hphantom{....}(Qubit[N], Qubit(), ... )) \{} \newline
                           \texttt{\hphantom{....}...} \newline
                           \texttt{\}} \\
  \end{keysref}

  \begin{keysref}{Debugging (quantum)}
    Print amplitudes \newline of wave function               & \texttt{DumpMachine("dump.txt")} \\
    Assert that a qubit is in $\ket{0}$ or $\ket{1}$ state   & \texttt{AssertQubit(Zero, \emph{zeroQubit})} \newline
                                                               \texttt{AssertQubit(One, \emph{oneQubit})} \\
  \end{keysref}

  \columnbreak

  \begin{keysref}{Basic Quantum Gates}
    Pauli gates          & \texttt{X(\emph{qubit})} : \newline
                           $\ket{0} \mapsto \ket{1}$, $\ket{1} \mapsto \ket{0}$ \newline
                           \texttt{Y(\emph{qubit})} : \newline
                           $\ket{0} \mapsto i \ket{1}$, $\ket{1} \mapsto -i \ket{0}$ \newline
                           \texttt{Z(\emph{qubit})} : \newline
                           $\ket{0} \mapsto \ket{0}$, $\ket{1} \mapsto -\ket{1}$ \\
    Hadamard             & \texttt{H(\emph{qubit})} : \newline
                           $\ket{0} \mapsto \ket{+} = \frac{1}{\sqrt{2}} ( \ket{0} + \ket{1} )$, \newline
                           $\ket{1} \mapsto \ket{-} = \frac{1}{\sqrt{2}} ( \ket{0} - \ket{1} )$ \\
    Controlled-NOT       & \texttt{CNOT(\emph{controlQubit}, \emph{targetQubit})} \newline
                           $\ket{00} \mapsto \ket{00}$, $\ket{01} \mapsto \ket{01}$, \newline
                           $\ket{10} \mapsto \ket{11}$, $\ket{11} \mapsto \ket{10}$ \\
    Apply several gates \newline (Bell pair example)  & \texttt{H(\emph{qubit1});} \newline 
                                                        \texttt{CNOT(\emph{qubit1}, \emph{qubit2});} \\
  \end{keysref}
 
  \begin{keysref}{Measurements}
    Measure qubit in Pauli $Z$ basis & \texttt{M(\emph{oneQubit})} \newline yields a \texttt{Result} (\texttt{Zero} or \texttt{One}) \\
    Reset qubit to $\ket{0}$     & \texttt{Reset(\emph{oneQubit})} \\
    Reset an array of \newline qubits to $\ket{0..0}$ & \texttt{ResetAll(\emph{register})} \\
  \end{keysref}

  \columnbreak

  \section{Working with \qs~from command line}

  \begin{keysref}{Command Line Basics}
    Change directory    & \texttt{cd \emph{dirname}} \\
    Go to home          & \texttt{cd \textasciitilde} \\
    Go up one directory & \texttt{cd ..} \\
    Make new directory  & \texttt{mkdir \emph{dirname}} \\
    Open current \newline directory in VS Code & \texttt{code .} \\
  \end{keysref}

  \begin{keysref}{Working with \qs~Projects}
    Create new project & \texttt{dotnet new console -lang Q\# --output \emph{project-dir}} \\
    Change directory to \newline project directory & \texttt{cd \emph{project-dir}} \\
    Build project                     & \texttt{dotnet build} \\
    Run all unit tests                & \texttt{dotnet test} \\
  \end{keysref}

\end{multicols}

\end{document} 
